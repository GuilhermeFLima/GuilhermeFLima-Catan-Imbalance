\documentclass[11pt]{article}
\usepackage{amsmath}
\usepackage{amsfonts}
\usepackage{multirow}
\usepackage{hyperref}
\title{The Variance of High Production Hexes in Catan}
\author{G.F. Lima}
\begin{document}
\maketitle
\section{Introduction}

Sometimes in a game of Catan the observed dice rolls can be very different from the expected distribution, as has been shown in \cite{DBD}. In order to give a more intuitive sense of this variance, we will focus on the particular case where, given two numbers with an equally high probability of occurring, during a game one rolls twice as many times as the other or more. We will also show that the chances of such occurrences are surprisingly high. Furthermore, we'll show that such issues can be mitigated by using balanced dice, as implemented in Colonist.io.


\section{Motivation}


A substantial component of Catan strategy is the placement of initial settlements. One of the most important features to optimize is production\footnote{There are obviously other features to take into account, such as synergy between the resources (e.g. wood + brick, ore + wheat + sheep), expansion spots, ports, and resource scarcity -- but these aren't the focus of this article.}, which is measured by the number of pips on each hex and is in proportional to the probability of each number being rolled. The choices made by players are guided by the expected production of each possible initial placement, yet in practice the dice outcomes can differ significantly from what's expected. For instance, it is not uncommon to have two players with equally good initial set-ups but with different high production numbers: one player could have settlements on a 6 and on a 5, while the  other has settlements on an 8 and on a 9. Both pairs have the same expected production, but we will show that the chances of one of these numbers rolling at least twice as many times as its counterpart during a game is roughly 37\%, even though they have equal probability of occurring. In contrast, this will happen in only $8\%$ of the games with the balanced dice featured in Colonist.io, and if we restrict ourselves to just the first half of a game, then the probability of such an imbalance goes as high as $66\%$ with random dice, while the balanced dice remain at $8\%$.


%Suppose you and one of your opponents have almost the exact same initial set up: you both have a good production and fair amount of whichever resource you feel is most important. Let's say, for example, that you both have five pips of ore. But there's one difference: your ore in on a 6 hex, while theirs is on an 8 hex. This seems balanced, since the 6's and 8's have the same probability of rolling, therefore they have the same expected number of rolls throughout the game. As much as that is true, it still doesn't account for the variance in each game: what if the 8's happen to roll twice as many times as the 6's? Was that a fair game? What are the odds of that happening? Or maybe you both have the same wheat production, say 4 pips, but you're on a 5 hex and they're on a 9 hex. Even if the 6's and the 8's roll the same number of times, if the 9's roll twice as many times as the 5's throughout the entire game, was that game fair?

% We intend to show that, with normal random dice, the probability of such an imbalance happening is actually non-trivial, in fact, it's around 37\%, i.e. roughly four out of ten games will feature this amount of variance among high production hexes with the same amount of pips. 

%Let's make things a bit more precise by defining our terms. If one of the 5-pip numbers rolls at least twice as many times as the other, I'll call that a \emph{5-pip-imbalance} (e.g. the 8's rolled twice as many times as the 6's...). Similarly, a situation where the 9's roll at least twice as many times as the 5's, or vice-versa, is a \emph{4-pip-imbalance}. 


\section{Numerical Simulation}

Given a sequence of dice rolls, and given two specific dice roll results $m$ and $n$, both having the same probability of occurring (e.g. 6 and 8) we shall say that there's \emph{high imbalance between m and n} if either $n$ occurs at least twice as many times as $m$ in the sequence, or  $m$ occurs at least twice as many times as $n$ in the sequence.

First, we shall assume that the average Catan game has roughly 70 turns. This seems in line with our personal experience of over 1000 games and the data gathered in \cite{BGB}. We simulated 1000000 (one million) games with 70 dice rolls for both random and balanced dice. Our original Python code can be found in \cite{githubsim}.

With random dice, the probability of high imbalance between 6 and 8 was 0.172698, or roughly 17\% and the probability of high imbalance between 6 and 8 or 5 and 9 was 0.365337, or roughly 37\%. On the other hand, with balanced dice the probability of high imbalance between 6 and 8 was 0.027685 $\cong 3\%$, and the probability of high imbalance between 6 and 8 or 5 and 9 was 0.079603 $\cong 8\% .$


\section{Analytical Solution}
If we wanted the probability of a number being rolled $k$ times in an average game of $n$ turns, we'd use the probability mass function of a binomial distribution:

$$
f(k, p, n) = \binom{n}{k}p^k(1-p)^{n-k}
$$
So, for example, the probability of a 7 being rolled exactly 20 times in a game with 70 turns is

$$
f(20, 70, 1/6) = \binom{70}{20} \times \left(\frac{1}{6}\right)^{20} \times \left(\frac{5}{6}\right)^{50}
$$
since the probability of rolling a 7 is 1/6.

The derivation of this probability function isn't too hard. The term $p^k$ represents the probability of $k$ independent events with individual probability $p$, and the term $(1-p)^{n-k}$ corresponds to the other $n-k$ events which must have probability $1-p$. The binomial coefficient represents the different ways these different events can be scrambled around.

The binomial distribution is based on the assumption that each trial is a Bernoulli trial, i.e. there are only two possible outcomes, which are usually refereed to as \emph{success} and \emph{failure}. To tackle our problem theoretically, we need a modified version of the binomial distribution where instead of Bernoulli trials with only two possible outcomes, we have trials with \emph{three} possible outcomes. Let's start with the case were we have three possible outcomes (e.g. an 8 rolls, a 6 rolls, or neither of these). Let's call the first outcome a \emph{1-success}, the second outcome a \emph{2-success}, and the third, a \emph{failure}. The formula for having $k_1$ 1-successes and $k_2$ 2-successes in $n$ trials is:

$$
f(k_1, p_1, k_2, p_2, n) = \frac{n!}{k_1! k_2! (n- k_1 - k_2)!} \quad p_1^{k_1} p_2^{k_2} (1 - p_1 - p_2)^{n - k_1 - k_2}
$$

We can further generalise to the case where we have $m$ different types of successes, and so we have finite sequences $(k_i)_1^m = k_1, \ldots, k_m$ and $(p_i)_1^m = p_1, \ldots, p_m$, and a probability mass function
$$
f((k_i)_1^m, (p_i)_1^m, n) = \frac{n!}{(\prod_1^m k_i!) (n- \sum_1^m k_i)!} \quad \left(\prod_1^m p_i^{k_i}\right)\left(1 - \sum_1^m p_i\right)^{\left(n - \sum_1^m k_i\right)}
$$

\subsection{Example:}
Knowing that the probability of an 8 rolling is the same as the probability of a 6 rolling, which is 5/36, then the probability of an 8 rolling 10 times and a 6 rolling 20 times in a game with 70 turns is given by

$$
f(10, 5/36, 20, 5/36, 70) =  \frac{70!}{(10!) (20!) (40!)} \times \left(\frac{5}{36}\right)^{10} \times \left(\frac{5}{36}\right)^{20} \times \left(\frac{26}{36}\right)^{40}
$$

Coming back to our original problem, we can now see that the probability of a 6 rolling at least twice as many times as an 8 in a game with 70 turns is given by
$$
\sum_{(k_1, k_2) \in \mathbf{K}} f(k_1, p, k_2, p, n)
$$
where $p$ = 5/36, $n=70$, and 
$$
\mathbf{K} = \{ (k_1, k_2) \in \mathbb{N}^ 2 \ | \ k_1 \geq 2\cdot k_2,\ \ k_1 + k_2 \leq n, \ \ k_1 > 0 \}
$$
 
By symmetry, this is the same probability of an 8 rolling at least twice as many times as a 6, and so the probability of either the 6's rolling at least twice as many times as the 8's, or the 8's rolling at least twice as many times as the 6's in a game with 70 turns is given by

$$
2 \cdot \sum_{(k_1, k_2) \in \mathbf{K}} f(k_1, 5/36, k_2, 5/36, 70) \cong 0.1725
$$
Similarly, the probability of an imbalance between the  5 and the 9 is given by the same formula except with $p=4/36 = 1/9$:

$$
2 \cdot \sum_{(k_1, k_2) \in \mathbf{K}} f(k_1, 1/9, k_2, 1/9, 70) \cong 0.2321
$$

To calculate the probability of high imbalance between 6 and 8 or high imbalance between 5 and 9, we need to use the  inclusion-exclusion principle, since the two cases have non-trivial intersection. In order to do this, we must first calculate the probability of imbalance between 6 and 8 \emph{and} imbalance between 5 and 9, and this is  

$$
4 \cdot \sum_{(k_1, k_2, k_3, k_4) \in \mathbf{L}} f((k_i)_1^4, (p_i)_1^4, n) \cong 0.039
$$

where $p_1 = p_2 = 5/36$, $p_3 = p_4 = 1/9$, $n=70$ and

$$
\mathbf{L} = \{ (k_1, k_2, k_3, k_4) \ | \ \ k_1 \geq 2\cdot k_2,\ \ k_3 \geq 2\cdot k_4,\ \ k_1 + k_2 + k_3 + k_4 \leq n, \ \ k_1 > 0, \ \ k_3 > 0 \}. 
$$
We are multiplying by 4 since, if $(k_1, k_2, k_3, k_4) \in \mathbf{L}$ then $(k_2, k_1, k_3, k_4)$, $(k_1, k_2, k_4, k_3)$, and  $(k_2, k_1, k_4, k_3)$ also represent situations with the same type of imbalance.

Hence we conclude that the approximate probability of either high imbalance between 6 and 8 or high imbalance between 5 and 9 is approximately $0.1725 + 0.2321 - 0.0390 = 0.3656$.

And so we see that our simulation adequately approximates the analytical solution. The Python code used for the numerical calculations may be found in \cite{githubsim}.

\section{Summary of Results}
The following table that summarizes and compares the balanced dice and the random dice. For each type of dice we used simulations of one million games.

\begin{center}
\vspace{1cm}
%\begin{table}
%\begin{figure*}

	\begin{tabular}{cc|c|c|}
\cline{3-4}
&& \multicolumn{2}{|c|}{Probability of high} \\
&& \multicolumn{2}{|c|}{imbalance between} \\
\cline{3-4}
&& 6 and 8 & 6 and 8, or \\
&& & 5 and 9 \\
\hline
\multicolumn{1}{|c}{\multirow{2}{*}{Random}} & 
\multicolumn{1}{|c|}{35 turns} & 37\% & 66\% \\
\cline{2-4}
\multicolumn{1}{|c}{\multirow{2}{*}{}} 
& \multicolumn{1}{|c|}{70 turns} & 17\% & 37\% \\
\hline
\multicolumn{1}{|c}{\multirow{2}{*}{Balanced}} & 
\multicolumn{1}{|c|}{35 turns} & 3\% & 8\% \\
\cline{2-4} 
\multicolumn{1}{|c}{\multirow{2}{*}{}} 
& \multicolumn{1}{|c|}{70 turns} & 3\% & 8\% \\
\hline
\end{tabular}
%\end{figure*}
%\end{table}
\end{center}
Although the imbalance probability is much higher when there are fewer turns, the robber mechanism can counter-balance early leads.


\begin{thebibliography}{9}

\bibitem{DBD}
  Jeff d'Eon, \textit{Designing Balanced Dice}, 2020.\\
  Colonist Blog: \\
  \url{<https://blog.colonist.io/designing-balanced-dice/>}
 
 \bibitem{BGB}
 Alex Cates, \textit{Board Game Breakdown: Settlers of Catan, the basics}, 2021. \\
 \url{<https://www.alexcates.com/post/board-game-breakdown-settlers-of-catan-the-basics>}

\bibitem{githubsim}
\url{<https://github.com/GuilhermeFLima/GuilhermeFLima-Catan-Imbalance/blob/main/imbalance-simulation.py>}

\bibitem{githubana}
\url{<https://github.com/GuilhermeFLima/GuilhermeFLima-Catan-Imbalance/blob/main/imbalance-analytical.py>}


\end{thebibliography}

\end{document}
